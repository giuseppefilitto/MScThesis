\documentclass{standalone}
\begin{document}
\section{Implementation}

I have implemented the pipeline by using python, which is an high level object oriented programming language. 
The supported python versions are: $3.6| 3.7| 3.8| 3.9$.
To perform operations on images (image filtering, input-output operations, etc\dots), I've used different libraries depending on the specific purpose, as in some cases anticipated in the previous section.
\\
The whole code is open-source and available on GitHub \cite{img-segm} and the relative documentation, made by using Sphinx \cite{Sphinx}, is available at: \url{https://img-segm.readthedocs.io/en/latest/?badge=latest}.
The installation is managed by setup.py, which also provides the full list of dependencies.
The pipeline installation is tested on MacOS (base environment) and on Linux by using the TravisCI host.
\\
The pipeline implementation provides also modules that allow one to load, visualize, processing the DICOM series and to train a U-Net model and provides scripts to handle DICOM series from command line. 
\\
The detailed description of each module and script is available on GitHub.
Once you have installed it, you can start to segment the images directly from your shell, passing as input the path of the directory containing the DICOM series, to obtain the prediction of response.
\\
There are different output options that will be described in the \textit{Results} chapter.\\
This section will be aimed at the implementation of the main steps of the pipeline.

\end{document}