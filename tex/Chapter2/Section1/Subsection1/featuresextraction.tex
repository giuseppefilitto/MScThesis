\documentclass{standalone}
\begin{document}
\subsection{Features Extraction}

This step consists of the extraction of the radiomic features from the images.
For this purpose the MRI scans of each patient and the segmented images are saved, by using \textsc{SimpleITK} library \cite{SimpleITK}, as 3D images in a particular format (.nrrd) required by \textsc{Pyradiomics} library \cite{Pyradiomics} to extract features from every single slice of the patient. 
From each patient, a total of 100 radiomic features were extracted.
The extraction settings were stored in a $\mathtt{Params.yaml}$ file required by \textsc{Pyradiomics}.
The data were then stored in a data frame containing 100 columns (one for each feature) and 48 rows (one for each patient) ready for the analysis step.


\end{document}