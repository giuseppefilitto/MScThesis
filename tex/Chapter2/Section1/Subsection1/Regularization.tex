\documentclass{standalone}
\begin{document}
\subsection{Regularization methods}
Regularization is a process used to prevent overfitting or to solve an ill-posed problem.
Regularization includes various methods.

\paragraph{Dropout}
consists in randomly ignoring individual nodes, with a probability $p$, at each training stage for reducing overfitting.
Dropout seems to reduce node interactions, leading them to learn more robust features that better generalize to new data\cite{wiki:cnn}.

\paragraph{Data Augmentation}
is a technique used to increase the number of data by adding slightly modified copies of already existing data or newly created synthetic data from existing data.
It acts as a regularizer since it helps in reducing overfitting\cite{wiki:cnn}.

\end{document}