\documentclass{standalone}
\usepackage{xr}
\externaldocument{../Chapter2/intro}
\begin{document}
\subsection{Segmentation}
Once trained, the CNN model was used for the segmentation of the MRI scans of each patient.
As mentioned previously, before the segmentation, the scans are pre-processed to reduce noise, for the gamma correction, and to check if the image size is $512 \times 512$.
During the pre-processing, as we know, the scans are also rescaled to binary floating-point 32-bit, required to work with \textsc{TensorFlow}, since the model is trained on images of that kind.
\\
The segmentation is done for each stack of slices of the patient using the trained CNN model to obtain the prediction.
\\
Then, using \textsc{OpenCV}\cite{opencv_library} functions it is possible to obtain a segmented area like the one in Figure \ref{contoured}, where the red contour represents the border of the predicted tumor area.


\begin{figure}[htp]

    \centering
    \includegraphics[width=.49\textwidth]{../images/BO56_10_cont.png}
    \includegraphics[width=.49\textwidth]{../images/BO85_7_cont.png}
    
    \caption{Images of colorectal cancer with identified tumor areas from two different patients. The red contour represents the border of the predicted tumor area.}
    \label{contoured}
    
    \end{figure}


\end{document}