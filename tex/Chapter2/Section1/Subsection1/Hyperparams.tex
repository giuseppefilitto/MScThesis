\documentclass{standalone}
\begin{document}
\subsection{Hyperparameters}

Hyperparameters consist in settings that are used to control the learning process of the network.

\paragraph{Kernel size}
is the dimension of the kernel matrix. Usually $3\times3$ or $2\times2$.

\paragraph{Padding}
consists in the addition of  0-valued pixels on the borders of an image.
This is done so that the border pixels are not lost from the output.

\paragraph{Stride}
consists in the number of pixels that the analysis window moves on each iteration. 
A stride of 2 means that each kernel is offset by 2 pixels from its predecessor.

\paragraph{Pooling size}
consists in the dimension of the pooling sub-region. Usually set $2\times2$.
Larger size can drastically reduce the dimension of the signal producing information loss.

\paragraph{Dilation}
consists in ignoring pixels within a kernel.
For example a dilation of 2 on a $3\times3$ kernel expands it to $7\times7$, while still processing 9 pixels.
\end{document}