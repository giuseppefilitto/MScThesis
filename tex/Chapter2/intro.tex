\documentclass{standalone}
\begin{document}
\chapter{Artificial Neural Networks}
Artificial Neural Networks (ANNs) are computational architectures derived from neural physiological models\cite{segmentationreview}.
An Artificial Neural Network is based on a collection of connected units or nodes called \textit{artificial neurons}. 
Each connection can transmit a signal to other neurons, processes and transmit it to other neurons. 
The output of each neuron is computed by some non-linear function of the sum of its inputs. 
The connections are called \textit{edges}. 
Neurons and edges typically have a \textit{weight}. 
The weight increases or decreases the strength of the signal at a connection. 
Typically, neurons are aggregated into \textit{layers}. 
Different layers may perform different transformations on their inputs. 
Signals travel from the \textit{input layer} (first layer), to the \textit{output layer} (last layer).
ANNs learn (or are trained) by processing examples, each of which contains a known \textit{input} and \textit{result}.
The difference between the former and the latter is called \textit{error}.
The network adjusts its weighted according to a learning rule and using this error valu.
Practically this can be done by defining a \textit{loss function} that is evaluated periodically during the learning process.
After a sufficient number of these adjustments the training can be terminated based upon certain criteria. 
This is process is also known as \textit{supervised learning}\cite{wiki:ann}.\\
Artificial Neural Networks (ANNs) have evolved into a broad family of techniques\cite{segmentationreview}.
For visual analysis are usually used Convolutional Neural Networks (CNNs) based on \textit{convolution kernels} or \textit{filters} that that extract feature maps\cite{wiki:cnn}.
Several architectures have been developed over the years.
In bio-medical image processing, the so-called U-Net\cite{unet}, is one of the most common architecture.\\
Since the main architecture of this project belongs to the CNNs class, the following section will be aimed to the description of Convolutional Neural Networks.
Then, the description of the main architecture.

\end{document}