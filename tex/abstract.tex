\documentclass{standalone}



\begin{document}

\begin{abstract}
Colorectal cancer is a malignant neoplasm of the large intestine resulting from the uncontrolled proliferation of one of the cells making up the colorectal tract. \\
Colorectal cancer is the second malignant tumor per number of deaths after lung cancer and the third per number of new cases after breast and lung cancer. 
Risk factors for this kind of cancer include colon polyps, long-standing ulcerative colitis, diabetes II, and genetic history (HNPCC or Lynch syndrome). 
In order to get information about a diagnosis, therapy evaluation on colorectal cancer, analysis on radiological images can be performed through the application of dedicated algorithms.\\
In this scenario, the correct and fast identification of the cancer regions is a
fundamental task. 
Up to now, this process is performed using manual or semi-automatic techniques, which are time-consuming and subjected to the operator's expertise.\\
The aim of this project is to develop and implement an automated pipeline to predict the response to neoadjuvant chemo-radiotherapy patients affected by colorectal cancer.
\end{abstract}

\end{document}