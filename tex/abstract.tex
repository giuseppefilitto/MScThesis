\documentclass{standalone}



\begin{document}

\begin{abstract}
Colorectal cancer is a malignant neoplasm of the large intestine resulting from the uncontrolled proliferation of one of the cells making up the colorectal tract. 
\\
In order to get information about diagnosis, therapy evaluation on colorectal cancer, analysis on radiological images can be performed through the application of dedicated algorithms.
Up to now, this process is performed using manual or semi-automatic techniques, which are time-consuming and highly operator dependent.
\\
The aim of this project is to develop and apply an automated pipeline to predict the response to neoadjuvant chemo-radiotherapy of patients affected by colorectal cancer.
Here, we propose an approach based on automatic segmentation and radiomic features extraction.
The segmentation process exploits a Convolutional Neural Network like U-Net, trained with medical annotations to perform the segmentation of the tumor areas.
Then, from the segmented regions, radiomic features are extracted and analyzed to obtain the prediction of response, based on the Tumor Regression Grade (TRG).
\\
We tested and developed our pipeline on MRI scans provided by the IRCCS Sant’Orsola-Malpighi Polyclinic.
The performance of the pipeline was measured for the segmentation purpose and for the prediction of response.
The results of these preliminary tests show that the pipeline is able to achieve a segmentation consistent with the medical annotations and a Dice Similarity Coefficient (DSC) coherent with literature.
Even for the prediction of response, the results show that the pipeline is able to correctly classify most of the cases.


\end{abstract}

\end{document}