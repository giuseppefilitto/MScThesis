\documentclass{standalone}
\begin{document}
\markboth{}{CONCLUSIONS}
\chapter*{Conclusions}\addcontentsline{toc}{chapter}{Conclusions}

In this work of thesis, I have developed and applied an automated pipeline to predict the response to neoadjuvant chemo-radiotherapy of patients affected by colorectal cancer.
Here, we proposed an approach based on automatic segmentation and radiomic features extraction.
\\
The pipeline was developed and tested on MRI scans provided by the IRCCS Sant'Orsola-Malpighi Policlinic.
The results of these preliminary tests show that the pipeline is able to segment the correct tumor regions, with the correct shape and contours, without interaction with trained personnel.
This was checked by comparing the segmentation results of the implemented pipeline with the manual annotations made by expert radiologists.
The implemented model was able to segment correctly even \textit{mucinous} cases, which are characterized by bright tumoral areas on MRI scans, different from the \textit{adenocarcinomas} characterized by dark tumoral areas.
Unfortunately, for some restricted cases it failed to segment correctly the images.
Among the possible issues, there might be the lack of data caused by the exclusion of some patients, due to the misregistration between the input images and the manual annotations. 
This could have affected the training phase and the ability of the model to generalize.
\\
The performance evaluation about the segmentation obtained by the pipeline on data coming from the validation set resulted in a DSC=0.71, that outperforms most of the ones in literature.
This is a significant result since from literature it comes out that the segmentation achieved by CNNs is quite hard to perform on colorectal cancer MRI scans.
The issues include the quality of data, medical annotations, and loss functions.
\\
From the segmented regions, radiomic features were extracted and analyzed to obtain the the prediction of response.
It is based on the Tumor Regression Grade (TRG) binarized into two main classes: Class 0 for complete response and Class 1 for moderate response.
\\
The results for the prediction of response show that the pipeline is able to classify correctly most of the cases.
The prediction presents high scores for classification metrics such as Precision, Recall, and F1-score.
In particular, the prediction is more accurate for Class 1 than Class 0 due to data imbalance towards Class 1.
The implemented pipeline was able to classify correctly 10 cases over 13 belonging to Class 0 and 
15 cases over 19 for Class 1.
\\
The performance of the classifier was also tested by computing the ROC curve, to evaluate the area under the curve (AUC).
The AUC resulted in $82 \%$ for both classes. 
Also after computing the average ROC curve, the AUC resulted in $84 \%$ and $85 \%$ by taking into account class imbalance (micro-average) and by giving the same weight to the classes (macro-average), respectively.
The AUC resulted in all the cases higher than the $80 \%$, greater than $50 \%$ which would correspond to a random classifier.
\\
The pipeline was implemented and developed by using the python language on the GitHub platform as an open-source project with the relative documentation.
The pipeline implementation is able to provide different outputs: the identified tumor area with red contours, the predicted binary mask of the segmented slice, the predicted probability map of each slice and the 3D mesh of the segmented areas.
\\
Further development of this project is possible, like embedding more information about the clinical status of the patient, genetic information, case history, etc... 
Moreover, increasing the number of patients in the analysis can give more general and robust results.
\\
In the end, this project provided a suitable approach with satisfactory results to segment MRI colorectal cancer scans, in order to predict the response to neoadjuvant chemo-radiotherapy by using radiomics features.






\end{document}