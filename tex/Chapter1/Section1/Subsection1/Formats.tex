\documentclass{standalone}
\begin{document}
    
\subsubsection{DICOM Format}

Image file formats provide a standard way to store information of an image in a computer file\cite{biondi}.
DICOM is the acronym of Digital Imaging and COmmunications in medicine.
It is not only a file format but also a network communication protocol\cite{Larobina}.
However here, we will discuss DICOM only as a medical image format.\\
DICOM file format establishes that the pixel data cannot be separated from the metadata\cite{Larobina}.
In other words, metadata and pixel data are merged in a unique file.
The header contains the description of the entire procedure used to generate the image in terms of acquisition protocol and scanning parameters\cite{Larobina}. 
It also contains patient information such as name, gender, age. 
For these reasons, the DICOM header is modality-dependent and varies in size. 
In practice, the header allows the image to be \textit{self-descriptive}.

\end{document}