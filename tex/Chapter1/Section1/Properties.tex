\documentclass{standalone}
\begin{document}
\section{General Properties}
The physical meaning of the image data depends on the performed image modality.
For example Computed Tomography (CT) and Magnetic Resonance Imaging (MRI), give structural information about the anatomy of the patient.
Other techniques, such as Positron Emission Tomography (PET) or Functional Magnetic Resonance Imaging (fMRI) give information about the functional properties of the patient's target organs. 
However, we can distinguish some general characteristics of digital images:

\paragraph{Pixel depth} is the number of bits used to encode the values of each pixel and it is related to the memory space used to store the amount of the encoded information\cite{Larobina}. 
Higher the number of bits, higher the information stored but also more memory space is required\cite{Larobina}. 
A group of 8 bits is called byte and represent the smallest quantity that can be stored in the memory of a computer.
For example, if an image has a pixel depth of 16 or 12 bits the computer will always store two bytes per pixel\cite{Larobina}.
With a pixel depth of 8 bits it is possible to codify and store integer numbers between 0 and 255 $(2^8-1)$.
There are also two formats for the encoding in binary of floating-point numbers: single precision 32-bit and the double precision 64-bit.

\paragraph{Pixel data} represent numerical values of the pixels are stored according to the data type.
Radiological images like CT and MR store 16 bits for each pixel as integers.
Image data may also be of complex type even if this data type is not common and can be bypassed by storing the real and imaginary parts as separate images.
For example, complex data is provided by arrays that in MRI store acquired data before the reconstruction (the so called k-space) or after the reconstruction if you choose to save both magnitude and phase images\cite{Larobina}.


\paragraph{Metadata} are information that describe the image stored usually at the beginning of the file as a header\cite{Larobina}. 
In the case of medical images, metadata have an important role due to the nature of the images itself.
For example, a magnetic resonance image will have parameters related to the pulse sequence used, timing information, number of acquisitions.
More, a PET image will have information about the radiopharmaceutical injected and the weight of the patient.
Medical image metadata can also include information about the patient.


\end{document}