\documentclass{standalone}
\begin{document}
\markboth{CHAPTER 1. MATERIALS AND METHODS}{1.5. PCA}
\section{Principal Component Analysis}
Principal component analysis (PCA) is a technique to reduce data dimensionality.
It replaces the $n$ original variables by a smaller number, $q$, of linear combinations, called principal components, of the original variables.
Its many application areas include data compression, image analysis, pattern recognition, regression and classification prediction\cite{PCA}.\\
The most common definition of PCA, due to Hotelling, is that, for a set of observed d-dimensional data vectors 
$ \{ \mathbf{t}_{n} \}$, $n \in \{1, \dots, N \} $, the $q$ principal axes $\mathbf{w}_j$, $j \in \{ 1, \dots, q \} $ are those orthonormal axes onto which the retained variance under projection is maximal.
It can be demonstrated that the vectors $\mathbf{w}_j$ are given by the $q$ dominant eigenvectors (i.e. those with the largest associated eigenvalues $\lambda$) of the sample covariance matrix $\mathbf{S} = \sum_{n}^{} ( \mathbf{t_n} - \mathbf{\bar{t}}) ( \mathbf{t_n} - \mathbf{\bar{t}})^T / N$ such that  $ \mathbf{S}\mathbf{w}_j = \lambda_j \mathbf{w_j}$ and where $\mathbf{\bar{t}}$ is the sample mean.
The vector $\mathbf{x}_{n} = \mathbf{W^T} (\mathbf{t_n} - \mathbf{\bar{t}})$, where $ \mathbf{W} = (\mathbf{w}_1 , \mathbf{w}_2 \dots \mathbf{w}_j)$ is thus a q-dimensional reduced representation of the observed vector $\mathbf{t}_{n}$ \cite{PCA}.
As we have mentioned, $\lambda_j$ is just the variance of each new feature dimension.
How to choose an appropriate $q$ depends on the Variance Contribution Rate  $\alpha_j = \lambda_j / \sum_{j}^{} \lambda_j $. 
This can be determined by looking at the cumulative explained variance ratio as a function of the number of components as shown in figure \ref{cumulativevr}.

\begin{figure}[ht]

    \centering
    \includegraphics[width=.7\textwidth]{../images/cumulative2.png}
    
    \caption{Example of cumulative explained variance ratio as a function of the number of components. This curve quantifies how much of the total variance is contained within the first N components. We can see that the first 10 components contain approximately 75 \% of the total variance, while to the reach the 100 \% you need around 50 components.}
    \label{cumulativevr}
    
    \end{figure}

\end{document}