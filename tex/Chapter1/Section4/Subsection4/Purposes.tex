\documentclass{standalone}
\begin{document}
\subsection{Possible Purposes Of Radiomics}

The possible applications of radiomics are based on a very wide range, from the prediction of clinical outcomes to the oncological diagnosis.
In this subsection, a brief overview of some general possible purposes will be given.
\subsubsection{Prediction of clinical outcomes} 
Radiomic features may be useful for predicting patient survival and describing intratumoral heterogeneity as demonstrated in a study by Aerts et al. \cite{Aerts}.
More, the usefulness of radiomics for predicting the immunotherapy response of patients with non-small cell lung cancer (NSCLC) using pretreatment CT and PET-CT images has been demonstrated by other studies\cite{tesicoppola}.
\subsubsection{Prediction of metastases}
Radiomic features can also predict the metastatic stage of tumors. 
For example, many radiomic features were identified as predictors of distant metastasis of lung adenocarcinoma in a study by Coroller et al.\cite{Coroller}.
They concluded that radiomic features may be useful in identifying patients at high risk of developing distant metastases, guiding clinicians in choosing the most effective treatment for individual patients.
\subsubsection{Prediction of physiological events}
Another possible application of radiomics analysis is the prediction of physiological events. 
Indeed, radiomics can be applied for the characterization and investigation of complex physiological events such as brain activity, which is usually studied with specific imaging techniques such as functional magnetic resonance ”fMRI”\cite{tesicoppola}. 


\end{document}