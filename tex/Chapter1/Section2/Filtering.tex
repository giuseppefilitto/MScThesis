\documentclass{standalone}
\begin{document}
\section{Spatial Domain Filtering}


Filtering is a procedure used for modifying or enhancing an image.
The value of any given pixel in the output image is determined by applying some operations to the neighborhood of the corresponding input pixel.
A pixel's neighborhood is some set of pixels, defined by their locations relative to that pixel.
The term \textit{spatial domain} indicates that the procedures operate directly on pixels.
Mathematically:

\begin{equation}
    g(x,y) = T[f(x,y)] 
\end{equation}

where $f(x, y)$ is the input image, $g(x, y)$ the output image and $T$ is an operator on $f$ defined over some neighborhood of $(x, y)$.
The operation on the point located in $(x, y)$ usually involves the application of a matrix called \textit{mask} or \textit{kernel}.
The application of the above-mentioned mask (or kernel) on an image is called \textit{spatial filtering}.
Filtering creates a new pixel with the same coordinates of the center of the neighborhood, whose value is the result of the operation.
For each $(x, y)$ of the image, the filter transform $g(x, y)$ is the linear combination of the mask coefficient $w(s, t)$ and the pixels of the image affected by the mask itself.
In general, we can write:
\begin{equation}
    g(x, y) = \sum_{s = -a}^{a} \sum_{t = -b}^{b} w(s, t) f(x + s, y + t)
\end{equation}  

\begin{figure}[ht]

    \centering
    \includegraphics[width=.9\textwidth]{../images/filtering.png}
    
    \caption{Example of spatial filtering. A filtered image is generated as the center of the mask or kernel, moves to every pixel in the input image. From \cite{filtering}}
    \label{filtering}
\end{figure}


\end{document}