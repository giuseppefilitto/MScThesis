\documentclass{standalone}
\begin{document}
\section{Spatial Domain Filtering}

Filtering is a technique for modifying or enhancing an image.
The term \textit{spatial domain} refers to the plane of the image itself, where the related processing methods are based on the direct manipulation of the pixels.
Among the various categories of spatial processing there are \textit{intensity transformations} and \textit{spatial filtering}.
The former operate on single pixels while the latter on every pixel's neighborhood.\\
Mathematically, we can express this processes as follow:

\begin{equation}
    g(x,y) = T[f(x,y)] 
\end{equation}

where $f(x, y)$ is the input image, $g(x, y)$ the output image and $T$ is an operator defined on $f$ around a point $(x, y)$.
The operation on the point located in $(x, y)$ usually involves the application of a matrix called \textit{mask} or \textit{kernel}.
It must have  $M \times N$ dimensions with M and N odd, in order to make the center of the mask coincide with the pixel in question, and occupy a small section of the image.
The application of the above-mentioned mask (or kernel) on an image is called \textit{spatial filtering}.  

\end{document}