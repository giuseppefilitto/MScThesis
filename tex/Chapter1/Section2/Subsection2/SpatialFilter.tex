\documentclass{standalone}
\begin{document}
\subsection{Spatial Filter}

A spatial filter consists in a (usually square) region called \textit{mask}  and a pre-defined operation applied on pixels of the region covered by the mask\cite{corrandconv}.
Filtering creates a new pixel with the same coordinates as the center of the neighborhood whose value is the result of the operation.
For each $(x, y)$ of the image, the filter transform $g(x, y)$ is the linear combination of the mask coefficient $w(s, t)$ and the pixels of the image affected by the mask itself.
\\
In general, we can write:
\begin{equation}
    g(x, y) = \sum_{s = -a}^{a} \sum_{t = -b}^{b} w(s, t) f(x + s, y + t)
\end{equation}  


\end{document}