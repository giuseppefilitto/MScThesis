\documentclass{standalone}
\begin{document}
\subsection{Methods}
During the year several segmentation methods have been developed\cite{biondi}.
There are several ways to classify these methods.
For example, depending if they require or not a training set of data, they can be classified into \textit{supervised} or \textit{unsupervised} methods.
More, they can be classified depending on the information type they use, like \textit{Pixel classification} methods, which use only information about pixel intensity, or \textit{Boundary following} methods which use edge information etc...\cite{biondi}.\\
Among the most common ones:

\paragraph{Thresholding}
is a very simple and common approach to segmentation.
This method is applied on the \textit{histogram} of the image.
The histogram of a digital image with intensity levels $L$ in the range $[0, \: L-1]$, is a discrete function $h(l_k) = n_k$ where $l_k$ is the k-th intensity value and $n_k$  is the number of pixels with intensity $l_k$.\\
Thresholding consists in binarizing an image through an (if) clause on the intensity value of each point after having determined a threshold value $T \in [0, \: L-1]$.
The threshold value $T$ is usually chosen by visual assessment on the image histogram but it can be automatize by algorithms like the \textit{Otsu algorithm}.
One drawback of this method is that some parts of the image can belong to the same class even if they belong to different objects.
In fact, thresholding does not take into account the spatial characteristics of the image.
Moreover, it is sensitive to noise and intensity inhomogeneity that corrupt the image histogram and make difficult the classification of pixels\cite{biondi}.

\begin{figure}[htp]

    \centering
    \includegraphics[width=.45\textwidth]{../images/thresholdhistogram.png}
    \includegraphics[width=.45\textwidth]{../images/thresholdexample.png}
    
    \caption{Example of thresholding segmentation using Fiji software\cite{Fiji}. \\
    \textit{ Left)} Image Histogram.\textit{ Right)} Result of thresholding.}
    \label{thresholding}
    
    \end{figure}


\paragraph{Artificial Neural Networks} 
are computational architectures derived from neural physiological models\cite{segmentationreview}.
Artificial Neural Networks (ANNs) have evolved into a broad family of techniques.
For visual analysis are usually used Convolutional Neural Networks (CNNs) based on \textit{convolution kernels} or \textit{filters} that slide along input data to extract feature maps\cite{wiki:cnn}.
Several architectures have been developed over the years, for different tasks and fields of application.
In bio-medical image processing, the so-called U-Net\cite{unet}, is one of the most common architecture.
U-Net is a kind of CNN which allows overcoming the requirement of many training data\cite{biondi, unet}.
However a better explanation of ANNs will be provided in the following chapter.




\end{document}