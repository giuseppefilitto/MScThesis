\documentclass{standalone}
\begin{document}
\section{Segmentation}
Image segmentation consists of the partitioning of an image into non-overlapping consistent regions that are homogeneous respect to some characteristics, such as intensity or texture\cite{biondi}.
The results of segmentation can be used to perform feature extraction, that provides fundamental information about organs or lesion volumes, to monitor the evolution of a particular disease and/or to evaluate the effects of therapeutical treatments etc...
Therefore, segmentation plays a crucial role for clinitians in identifying diseases such as tumors.
Segmentation, depending on the technique, can be manual, semi-manual or automatic:

\paragraph{Manual} is still the most reliable and precise method but it is time-consuming, highly operator-dependent and subject to operator expertise\cite{tesicoppola}.

\paragraph{Semi-Manual} is a faster method compared to the manual one and it is based on the traditional image processing methods such as thresholding and clustering. However, despite the time savings it is operator-dependent\cite{tesicoppola}.

\paragraph{Automatic} is the faster method compared to the other ones and it is not operator-dependent. However, the implementation of the algorithms is harder to perform\cite{tesicoppola}. 


\end{document}