\documentclass{standalone}
\begin{document}
\section{Segmentation}
Image segmentation is the partitioning of an image into non-overlapping consistent regions that are homogeneous to some characteristics, such as intensity or texture\cite{biondi}.
The results of segmentation can be used to perform image feature extraction, which provides fundamental information about organs or lesion volumes, to monitor the evolution of a particular disease, and/or to evaluate the effects of therapeutical treatment.
Therefore, segmentation plays a crucial role for clinicians in identifying diseases such as tumors.
Segmentation, depending on the technique, can be manual, semi-manual, or automatic:

\paragraph{Manual} techniques are still the most reliable and precise methods but they are time-consuming, highly operator-dependent, and subject to operator expertise\cite{tesicoppola}.

\paragraph{Semi-Manual} techniques are faster compared to manual ones.
They are based on traditional image processing methods such as thresholding and clustering. However, despite the time savings they are operator-dependent\cite{tesicoppola}.

\paragraph{Automatic} techniques are the faster methods compared to manual and semi-manual ones. Moreover, they are not operator-dependent. However, the implementation of the algorithms is harder to perform\cite{tesicoppola}. 


\end{document}