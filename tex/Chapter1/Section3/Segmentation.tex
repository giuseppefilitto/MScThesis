\documentclass{standalone}
\begin{document}
\section{Segmentation}

Image segmentation refers to the process of partitioning a digital image into multiple segments (i.e. set of pixels),  regions or objects so as to change the representation of an image into something that is more meaningful and easier to analyze. Pixels in a region are similar according to some homogeneity criteria such as colour, intensity or texture, so as to locate and identify objects and boundaries in an image\cite{imagesegmentation}.
\\
Image segmentation applications range from medical filed (i.e locate tumors and other pathologies, study of anatomical structure), object detection in satellite images (roads, forests, cars), face recognition, finger print recognition, etc...
\\
In general, segmentation, depending on the technique, can be manual, semi-manual, or automatic:

\subsubsection{Manual Segmentation}
Manual techniques are still the most reliable and precise methods but they are time-consuming, highly operator-dependent, and require expert operators\cite{tesicoppola}.
Examples of manual segmentation techniques include partitioning an image selecting manually the border of pixels belonging to a specific region or object (i.e tumors in medical images).

\subsubsection{Semi-Manual Segmentation}
Semi-Manual techniques are faster compared to manual ones.
They are based on traditional image processing methods such as thresholding and clustering. 
Despite the time savings they are still operator-dependent\cite{tesicoppola}.
Moreover, techniques such as thresholding do not take into account the spatial characteristics of the image so it is not always possible to get the desired partitioning.

\subsubsection{Automatic Segmentation}
Automatic segmentation involve faster techniques compared to manual and semi-manual ones. 
Moreover, they are not operator-dependent. 
Among them we find Convolutional Neural Networks.
However, the implementation of the networks ad algorithms is harder to perform\cite{tesicoppola}. 


\end{document}